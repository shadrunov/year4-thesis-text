\section{Security analysis of hardening techniques}

In this section, we will review previously reported vulnerabilities in the Linux kernel and container runtimes, in order to determine which methods of enhancing security are effective and which are not, in specific cases.

\subsection{Selection of Vulnerabilities}
In order to combine a list of security vulnerabilities that can lead to the execution of containerized applications, open sources on the Internet and literature were reviewed. Container Security Site provides a list of container security vulnerabilities compiled by Christophe Tafani-Dereeper, a cloud security expert \cite{s:christoph1}. This list includes the Linux CVEs and vulnerabilities in runc, containerd, Docker and Kubernetes. 

In literature, several researchers have mentioned container breakout vulnerabilities that we should also consider in our review. Jian and Chen in their work explore the possibility of bypassing container isolation through two vulnerabilities: CVE-2014-0038 and CVE-2016-5195. The first vulnerability, CVE-2014-0038, is a bug in the Linux kernel that allows local users to elevate privileges through a heap overflow. The second vulnerability, CVE-2016-5195, also known as the DirtyCOW vulnerability, exploits a race condition to overwrite arbitrary memory areas. This vulnerability could allow an attacker to execute arbitrary code or overwrite critical system files \cite{acm:5}. The researchers additionally mention other vulnerabilities found in Docker. These vulnerabilities exploit path traversals, chroot escapes and access to special file systems on the host to escape from the container. 

Bélair et al. in their study suggest examining CVE-2019-5736, a runc vulnerability. This vulnerability allows intruders to take control of the runc binary, and thus trigger the execution of arbitrary code on the host \cite{acm:6}.

Another study on the security of containerized applications, conducted by Reeves et al., identifies a list of potential vulnerabilities that could lead to container escape \cite{c:19}. In particular, the authors suggest that there are three main categories of vulnerabilities that can be exploited to escape from the container:
\begin{itemize}
    \item \textbf{File descriptor tampering}: CVE-2016-8649 (LXC), CVE-2016-9962 (runc), CVE-2019-5736 (runc);
    \item \textbf{Bypass access control}: CVE-2019-19921 (runc), CVE-2020-15257 (containerd), CVE-2020-2023, CVE-2020-2025, CVE-2020-2026 (kata);
    \item \textbf{Injection to host binaries}: CVE-2018-15664 and CVE-2019-14271 (Docker).
\end{itemize}

After reviewing the sources, we have compiled the information in the table \ref{tab:escapecve}.


\begin{table}[hbt]
    \centering
    \begin{tabular}{| p{0.18\linewidth} | p{0.77\linewidth} |} \hline
        \multicolumn{2}{|l|}{\textbf{Linux}} \\ \hline
        CVE-2022-0847 & Kernel bug allowing local users to change unwritable areas of memory (DirtyPipe) \\ \hline
        CVE-2022-0492 & Control groups v1 bug allowing an attacker to escape a container by rewriting release\_agent command \\ \hline
        CVE-2022-0185 & Buffer overflow vulnerability potentially leading to privilege escalation \\ \hline
        CVE-2021-3490 & Buffer overflow vulnerability in eBPF evaluation in the Linux kernel \\ \hline
        CVE-2021-31440 & Flaw in the way the kernel handles eBPF programs allowing privilege escalation \\ \hline
        CVE-2021-22555 & Heap out-of-bounds write vulnerability in Netfilter component of the kernel \\ \hline
        CVE-2017-1000112 & Memory corruption vulnerability in the way kernel handles UDP packets \\ \hline
        CVE-2017-5123 & Flaw in waitid system call allowing unprivileged users to bypass security sandboxes \\ \hline
        CVE-2016-5195 & Copy-on-write mishandling allowing attackers to write to a read-only memory areas (DirtyCOW) \\ \hline
        \multicolumn{2}{|l|}{\textbf{runc}} \\ \hline
        CVE-2024-21626 & A flow in the way runc handles file descriptors allowing a maliciously created container access host file system (Leaky Vessels) \\ \hline
        CVE-2021-30465 & Race condition in runc when mounting volumes into several specifically created containers allowing host file system access \\ \hline
        CVE-2019-19921 & Incorrect access control when mounting volumes in runc allowing an attacker to spawn two containers with specific volumes configuration to break out of the container \\ \hline
        CVE-2019-5736 & Runc vulnerability allowing an attacker to overwrite runc binary on the host and execute arbitrary code  \\ \hline
        CVE-2016-9962 & Ptrace between container processes allowed an access to process initialization artefacts leading to breakout \\ \hline
        \multicolumn{2}{|l|}{\textbf{containerd}} \\ \hline
        CVE-2022-23648 & Flaw in volume mounting allowed specially crafted image configuration to gain access to arbitrary files and directories on the host \\ \hline
        CVE-2020-15257 & Improper isolation of the containerd-shim API alowed malicious containers running in the same network namespace as the shim getting escalated privileges \\ \hline
        \multicolumn{2}{|l|}{\textbf{Docker}} \\ \hline
        CVE-2024-23653 & Vulnerability in BuildKit allowing attackers to bypass security restrictions from malicious image configuration \\ \hline
        CVE-2024-23651 & Vulnerability in BuildKit allowing for container escape when building an image using a malicious image configuration \\ \hline
        CVE-2021-21284 & Race condition allowing root user in container namespaces with some access to the host file system to modify configuration files and escalate to root user on the host \\ \hline
        CVE-2019-14271 & Malicious container can gain full root access on the host injecting code to the dynamically loaded library which loads to docker cp command \\ \hline
        CVE-2018-15664 & Flaw in docker cp command are allows attacker to copy arbitrary code to files on the host pointed by the symlink in container \\ \hline
    \end{tabular}
    \caption{Container breakout vulnerabilities}
    \label{tab:escapecve}
\end{table}

\FloatBarrier
\clearpage


\subsection{Linux vulnerabilities}
Next, we will discuss the Linux kernel's vulnerabilities and explore ways to exploit them, as well as the protection mechanisms that can be used to prevent exploitation.

\subsubsection{CVE-2022-0847}

\subsubsection*{Description}

CVE-2022-0847 (DirtyPipe) — Linux kernel vulnerability existed in versions from 5.8 to 5.16.11, 5.15.25 and 5.10.102. It is related to the implementation of access control for buffers. Attackers may exploit this vulnerability to gain write access to cached memory pages, even if they are marked as read-only. In the context of containers, an attacker could use this vulnerability to modify the read-only layers of a container image, potentially affecting other containers created from the same image \cite{s:h:2}.

Exploit published on the website of the vulnerability (\url{https://dirtypipe.cm4all.com}) works the following way. First, a process opens a file for reading, copying its contents to memory. Next, an empty buffer is created and filled with random bytes, so all pages can be marked with the \texttt{PIPE\_BUF\_FLAG\_CAN\_MERGE} flag. Then, the \texttt{splice()} function is called to point the buffer to the cached file pages. After that, the attacker writes a payload to the buffer, which will be written to the file cache because the flag allows writing to existing pages. As a result, other processes that access the file will get a tampered version from the cache. 

\subsubsection*{Defence}

To prevent this vulnerability from being exploited, seccomp profiles can be used. It is sufficient to disable the \texttt{splice()} system call, which is necessary for moving data between a file descriptor and a buffer, as discussed in \cite{s:h:4}. With this call disabled, the exploit described above will not be able to complete its task. In fact, this call is blocked by the default seccomp filter enforced by Docker.

Other defense mechanisms do not prevent the exploitation of CVE-2022-0847. While setting a user ID can limit the number of files that can be accessed for this attack, it is not enough to prevent it. The exploit requires read permissions to the target file, and Linux file permissions can prevent regular users from reading sensitive files. However, there are still many sensitive system files that can be read by anyone. Also, dropping the capabilities has no effect on the exploitability of the vulnerability. AppArmor and SELinux may be able to limit the range of files that a user can access, but they also do not prevent exploitation.

It is also clear that alternative runtimes that use virtual machine isolation for containers can successfully mitigate this vulnerability. This is because the container uses its own kernel, which prevents it from affecting the host.

This is summarised in Table \ref{tab:h:1}.

\begin{table}[H]
    \centering \small
    \begin{tabular}{| p{0.18\linewidth} | p{0.1\linewidth} | p{0.1\linewidth} | p{0.1\linewidth} | p{0.1\linewidth} | p{0.12\linewidth} | p{0.12\linewidth} |} \hline
    CVE & Rootless mode & User ID & Dropped Privileges & seccomp & AppArmor, SELinux & Alternative Runtimes \\ \hline
    CVE-2022-0847 & - & - & - & \cellcolor{green!25} + & - & \cellcolor{green!25} +  \\ \hline
    \end{tabular}
    \caption{CVE-2022-0847}
    \label{tab:h:1}
\end{table}


\subsubsection{CVE-2022-0492}

\subsubsection*{Description}

is a security vulnerability in the Linux kernel that affects control groups version 1 (cgroups). This issue could allow an attacker to escape from a containerized environment and access the host system. The bug is caused by the \texttt{release\_agent} file in cgroups v1. This file is executed each time when the last process leaves the cgroup and the group becomes empty. Since the command in the \texttt{release\_agent} is executed with elevated permissions on the host, tampering with this file could lead to a host compromise.

The exploit for CVE-2022-0492 works the following way. It is necessary to be assigned to the namespace where the process has \texttt{CAP\_SYS\_ADMIN} capability (this privilege could be granted, for instance, by \texttt{--privileged} flag when container is started by Docker or by switching to a new User namespace. This is applicable to systems with \texttt{kernel.unprivileged\_userns\_clone} kernel parameter set to 1). With this capability, the attacker has to mount \texttt{cgroupfs} (example command: \texttt{mount -t cgroup -o rdma cgroup /mnt}). Next, new cgroup must be created (\texttt{mkdir /mnt/x}) and release agent enabled (\texttt{echo 1 > /mnt/x/notify\_on\_release}). Also, the desired command should be written to the \texttt{/mnt/release\_agent} file. Final step is to spawn and terminate a process in the freshly created control group \texttt{x} (\texttt{sh -c "echo \$\$ > /mnt/x/cgroup.procs"}). After that the command specified in \texttt{/mnt/release\_agent} file will be performed on the host with root privileges. The process of exploitation is illustrated in Figure \ref{img:h1}.

\image{h1.png}{Exploitation of CVE-2022-0492}{img:h1}{1}


\subsubsection*{Defence}
Let us consider the requirements for successfully executing the exploit. Firstly, the cgroups version 1 must be enabled on the host, which is an obsolete version and is being gradually replaced by version 2. Next, the user inside the container must be root, otherwise the \texttt{cgroupfs} file system will not be accessable. Also, either the \texttt{CAP\_SYS\_ADMIN} capability is required or the ability to obtain it by switching to a new user namespace. Next, the containerized process needs a capability to perform mounting. Fortunately, it is possible to mitigate the exploitation on every stage of the process.

Rootless mode (using a user namespace to map a root user inside a container to a non-root user on the host) as well as setting the User ID inside a container makes it impossible to access \texttt{cgroupfs} and therefore exploit CVE-2022-0492, because this file system is only accessible by the root user\footnote{\url{https://youtu.be/M4a2b4KkXN8}}. Linux security modules such as AppArmor and SELinux restrict the mounting of volumes to containers at runtime, protecting the system from exploitation, because \texttt{cgroupfs} must be mounted inside the container in write mode\footnote{\url{https://unit42.paloaltonetworks.com/cve-2022-0492-cgroups}}. Dropping privileges (including \texttt{CAP\_SYS\_ADMIN}) effectively prevents the attack as long as switching to a new user namespaces for non-root users is prohibited (with kernel setting \texttt{kernel.unprivileged\_userns\_clone} set to 0). Finally, even the default seccomp profile enforced by Docker disables the \texttt{mount()} and \texttt{unshare()} system calls required for mounting \texttt{cgroupfs} and switching to a new namespace\footnote{\url{https://docs.docker.com/engine/security/seccomp}}. For this reason, the default seccomp profile must be disabled in order to successfully carry out the attack. Overall, it is clear that several layers of additional defense mechanisms have been proven to be effective against the CVE-2022-0492 vulnerability.

VM-based container runtimes can also successfully mitigate this vulnerability, as it requires the container to be able to access host kernel and host objects. It was also stated by Google that gVisor is unaffected by this vulnerability as it prevents access to the vulnerable system call on the host\footnote{\url{https://cloud.google.com/anthos/clusters/docs/security-bulletins}}.

The summary of effectiveness of security tools is presented in Table \ref{tab:h:2}.

\begin{table}[H]
    \centering \small
    \begin{tabular}{| p{0.18\linewidth} | p{0.1\linewidth} | p{0.08\linewidth} | p{0.12\linewidth} | p{0.09\linewidth} | p{0.12\linewidth} | p{0.12\linewidth} |} \hline
    CVE & Rootless mode & User ID & Dropped Privileges & seccomp & AppArmor, SELinux & Alternative Runtimes \\ \hline
    CVE-2022-0492 & \cellcolor{green!25} + & \cellcolor{green!25} + & \cellcolor{yellow!25} +/- \linebreak if no userns & \cellcolor{green!25} + & \cellcolor{green!25} + & \cellcolor{green!25} + \\ \hline
    \end{tabular}
    \caption{CVE-2022-0492}
    \label{tab:h:2}
\end{table}

\subsubsection{CVE-2022-0185}
\subsubsection*{Description}

CVE-2022-0185 is a kernel vulnerability that affects the kernel versions from 5.1 to 5.16. This vulnerability exploits a buffer overflowing in the component of kernel called file system Context. Due to an error in \texttt{legacy\_parse\_param} function, the length of arguments supplied was not checked, allowing an unprivileged user (if it is permitted to switch to new user namespace for regular users) or the user with \texttt{CAP\_SYS\_ADMIN} capability in the current namespace to elevate the privileges in any other namespace including root one, thus breaking out of a container. The privileges are required to open a special file system which lacks the support of the Kernel file system Context API and this way trigger the fallback to the legacy function. The exploits for this vulnerability can be found on GitHub\footnote{\url{https://github.com/Crusaders-of-Rust/CVE-2022-0185}}.

\subsubsection*{Defence}

CVE-2022-0185 vulnerability uses the kernel bug, and in such cases it is likely that kernel security mechanisms will have no effect (Table \ref*{tab:h:3}).

The main requirement for exploiting this vulnerability is having the  \texttt{CAP\_SYS\_ADMIN} capability. Rootless mode and settig User ID do not prevent the exploitation of this vulnerabilityif switching to new user namespaces for non-root users is not prohibited \linebreak (\texttt{kernel.unprivileged\_userns\_clone = 0}). Otherwise, a user may simply run a process in a new user namespace, where they will be granted the \texttt{CAP\_SYS\_ADMIN} privilege successfully. Dropping capabilities (\texttt{CAP\_SYS\_ADMIN}) is also effective under the conditions mentioned above.

The recommended way to mitigate this vulnerability is by using a seccomp profile, in case it is not possible to patch the kernel. The default Docker seccomp profile filters the \texttt{unshare} system call which is needed to obtain the required privileges in a new namespace. Therefore, this way, the vulnerability cannot be exploited\footnote{\url{https://www.aquasec.com/blog/cve-2022-0185-linux-kernel-container-escape-in-kubernetes}}.

Finally, alternative container runtimes again provide a solution to this vulnerability by providing a separate kernel for the container process.

\begin{table}[H]
    \centering \small
    \begin{tabular}{| p{0.18\linewidth} | p{0.1\linewidth} | p{0.08\linewidth} | p{0.12\linewidth} | p{0.09\linewidth} | p{0.12\linewidth} | p{0.12\linewidth} |} \hline
    CVE & Rootless mode & User ID & Dropped Privileges & seccomp & AppArmor, SELinux & Alternative Runtimes \\ \hline
    CVE-2022-0185 & - & - & \cellcolor{yellow!25} +/- \linebreak if no userns & \cellcolor{green!25} + &  & \cellcolor{green!25} + \\ \hline
    \end{tabular}
    \caption{CVE-2022-0185}
    \label{tab:h:3}
\end{table}


\subsubsection{CVE-2021-3490}
\subsubsection*{Description}

CVE-2021-3490 is a vulnerability in the Linux kernel that has been present since version 5.7. This issue is caused by a lack of proper checking of the bounds of 32-bit values when executing bitwise logic operations such as AND, OR, and XOR within the eBPF kernel module. A malicious actor could exploit this vulnerability to access data beyond the intended memory boundaries and execute malicious code through the kernel. If a process inside a container has the \texttt{CAP\_BPF} capability, this vulnerability could lead to container escape.


\subsubsection*{Defence}

For successful exploitation of this vulnerability and escaping from the container constraints, the isolated process must be granted additional privileges: \texttt{CAP\_BPF} or \linebreak \texttt{CAP\_SYS\_ADMIN}. In addition, the \texttt{bpf()} system call must be available to the process. These two requirements explain the effectiveness of dropping privileges (for instance, the \linebreak \texttt{--privileged} flag in Docker should not be set) as well as the default seccomp filter which disables \texttt{bpf()} system calls for running containers are effective. Other kernel security features cannot be used in the case of this vulnerability\footnote{\url{https://www.crowdstrike.com/blog/exploiting-cve-2021-3490-for-container-escapes}}.

Another way to mitigate this vulnerability is to use non-kernel virtualization provided by virtual machine-based runtimes (Table \ref{tab:h:4}).


\begin{table}[H]
    \centering \small
    \begin{tabular}{| p{0.18\linewidth} | p{0.1\linewidth} | p{0.08\linewidth} | p{0.12\linewidth} | p{0.09\linewidth} | p{0.12\linewidth} | p{0.12\linewidth} |} \hline
    CVE & Rootless mode & User ID & Dropped Privileges & seccomp & AppArmor, SELinux & Alternative Runtimes \\ \hline
    CVE-2021-3490 & - & - & \cellcolor{yellow!25} +/- \linebreak if no userns & \cellcolor{green!25} + &  & \cellcolor{green!25} + \\ \hline
    \end{tabular}
    \caption{CVE-2021-3490}
    \label{tab:h:4}
\end{table}



\subsubsection{CVE-2021-31440}
\subsubsection*{Description}

CVE-2021-31440 is a Linux kernel vulnerability that relates to eBPF (extended Berkeley Packet Filter) processing. This vulnerability affects kernel versions 5.7 and above. The cause of the vulnerability is the lack of proper validation for user-supplied eBPF programs before they are executed. As a result, a specially crafted payload can be used to allow a local user to gain elevated privileges and execute arbitrary code within the kernel.

\subsubsection*{Defence}

The exploitation scenario is similar to the vulnerability discussed previously, so the conclusions remain correct. Effective defense measures include limited privileges and seccomp filters. In this way, a process will not be able to access the eBPF features of the kernel (Table \ref{tab:h:5}).

\begin{table}[H]
    \centering \small
    \begin{tabular}{| p{0.18\linewidth} | p{0.1\linewidth} | p{0.08\linewidth} | p{0.12\linewidth} | p{0.09\linewidth} | p{0.12\linewidth} | p{0.12\linewidth} |} \hline
    CVE & Rootless mode & User ID & Dropped Privileges & seccomp & AppArmor, SELinux & Alternative Runtimes \\ \hline
    CVE-2021-31440 & - & - & \cellcolor{yellow!25} +/- \linebreak if no userns & \cellcolor{green!25} + &  & \cellcolor{green!25} + \\ \hline
    \end{tabular}
    \caption{CVE-2021-31440}
    \label{tab:h:5}
\end{table}




\subsubsection{CVE-2021-22555}
\subsubsection*{Description}

CVE-2021-22555 is a Linux kernel vulnerability that has remained undiscovered for a long time, since the version 2.6.19. It resides in the Netfilter subsystem of the kernel and allows local users to obtain root privileges, even from inside a container\footnote{\url{https://github.com/google/security-research/blob/master/pocs/linux/cve-2021-22555/writeup.md}}.

The bug is related to the Netfilter handlers \texttt{IPT\_SO\_SET\_REPLACE} and \linebreak \texttt{IP6T\_SO\_SET\_REPLACE} leading to a buffer overflow and subsequent execution of arbitrary instructions with elevated privileges.

\subsubsection*{Defence}

To access the flawed Netfilter handlers, the process has to obtain the \texttt{CAP\_NET\_ADMIN} capability which generally allows privileged operations on the network stack. To gain this privilege, it is sufficient to switch to new user and network namespaces. For this reason, dropped privileges provide protection from the exploitation only if switching to namespaces is disabled for non-root users. If this setting is disabled, dropping privileges is an effective way to defend the system. 

Seccomp profiles are effective in the context of CVE-2021-22555 as well. The default profile set by Docker bans the \texttt{unshare()} system call and \texttt{clone()} system call required to switch to new namespaces and obtain missing capabilities. 

Finally, VM-based container runtimes are also effective in mitigation of this vulnerability (Table \ref{tab:h:6}).

\begin{table}[H]
    \centering \small
    \begin{tabular}{| p{0.18\linewidth} | p{0.1\linewidth} | p{0.08\linewidth} | p{0.12\linewidth} | p{0.09\linewidth} | p{0.12\linewidth} | p{0.12\linewidth} |} \hline
    CVE & Rootless mode & User ID & Dropped Privileges & seccomp & AppArmor, SELinux & Alternative Runtimes \\ \hline
    CVE-2021-22555 & - & - & \cellcolor{yellow!25} +/- \linebreak if no userns & \cellcolor{green!25} + &  & \cellcolor{green!25} + \\ \hline
    \end{tabular}
    \caption{CVE-2021-22555}
    \label{tab:h:6}
\end{table}



\subsubsection{CVE-2017-1000112}
\subsubsection*{Description}

CVE-2017-1000112 is a vulnerability that is caused by a race condition in the process of converting UDP segments into IP packets. This process is called UFO (UDP Fragmentation Offload), and it is a way to delegate the task of breaking large UDP datagrams into smaller packets to the network interface card. Currently, this deprecated kernel feature allows an attacker to switch the type of object from UFO to non-UFO, changing the size of the memory required by the object. As a result, the attacker gains access to arbitrary memory areas and can overwrite the callback function pointer, causing malicious code to be executed and escalating the privileges.

\subsubsection*{Defence}

To successfully exploit this vulnerability the process needs the \texttt{CAP\_NET\_ADMIN} capability. Similarly to the previously discussed vulnerability, it is useful to drop the elevated privileges for the running process if the mechanism for switching to new namespaces is disabled. Also, seccomp profile disabling the \texttt{unshare()} and \texttt{clone()} system calls is effective as well (Table \ref{tab:h:7}).

\begin{table}[H]
    \centering \small
    \begin{tabular}{| p{0.18\linewidth} | p{0.1\linewidth} | p{0.08\linewidth} | p{0.12\linewidth} | p{0.09\linewidth} | p{0.12\linewidth} | p{0.12\linewidth} |} \hline
    CVE & Rootless mode & User ID & Dropped Privileges & seccomp & AppArmor, SELinux & Alternative Runtimes \\ \hline
    CVE-2017-1000112 & - & - & \cellcolor{yellow!25} +/- \linebreak if no userns & \cellcolor{green!25} + &  & \cellcolor{green!25} + \\ \hline
    \end{tabular}
    \caption{CVE-2017-1000112}
    \label{tab:h:7}
\end{table}


\subsubsection{CVE-2017-5123}
\subsubsection*{Description}

CVE-2017-5123 is a Linux kernel vulnerability in the \texttt{waitd} system call implementation. This system call is a function used by a parent process to wait for the state of its child process.

The bug is located in the \texttt{kernel/exit.c} allows the user to write data to an arbitrary memory address due to a lack of proper checking if the pointer is within the user address space. Existing exploits suggest overriding the permissions assigned by Docker, thus gaining additional privileges and potentially escaping the container.

The possible exploit for a container breakout works as follows. The \texttt{fork()} system call is called multiple times to create multiple structures in the kernel memory heap. Zero values are then written to addresses where these structures might be located. Finally, if any of the structures are overwritten (which can be seen by calling \texttt{getuid}), the rest of the structure can be overwritten and additional capabilities can be gained.

\subsubsection*{Defence}

This vulnerability in the \texttt{waitd} system call can be exploited very easily. The attacker needs an access to the \texttt{fork} and \texttt{getuid} system calls, which are typically granted to all users. No additional privileges or root access are required. The only possible protective measure would be to isolate the vulnerable system from the host kernel by running it in a container within a virtual machine (Table \ref{tab:h:8}).

\begin{table}[H]
    \centering \small
    \begin{tabular}{| p{0.18\linewidth} | p{0.1\linewidth} | p{0.08\linewidth} | p{0.12\linewidth} | p{0.09\linewidth} | p{0.12\linewidth} | p{0.12\linewidth} |} \hline
    CVE & Rootless mode & User ID & Dropped Privileges & seccomp & AppArmor, SELinux & Alternative Runtimes \\ \hline
    CVE-2017-5123 & - & - & - & - & - & \cellcolor{green!25} + \\ \hline
    \end{tabular}
    \caption{CVE-2017-5123}
    \label{tab:h:8}
\end{table}


\subsubsection{CVE-2016-5195}
\subsubsection*{Description}

CVE-2016-5195 vulnerability also known as DirtyCow affects the Linux kernel versions from 2.6.22 to 4.8.3. This vulnerability is is again related to the race condition in the implementation of copy-on-write mechanism in memory pages marked with \texttt{dirty bit} flag. An attacker with local access can exploit this vulnerability to elevate their privileges and modify any file, potentially leading to an escape from a container.

\subsubsection*{Defence}

The ways to mitigate CVE-2016-5195 are also limited as the only requirement for exploitation is to be able to read from the file. However, on some systems the most popular exploit may be limited by disabling the \texttt{ptrace} system call. This system call is required by one of the most popular exploits\footnote{\url{https://github.com/scumjr/dirtycow-vdso}} based on VSDO shared library. This shared object is unique system-wide and it is available for any provess to read. By overwriting this shared library a malicious process can escape from the container (Table \ref{tab:h:9}).

\begin{table}[H]
    \centering \small
    \begin{tabular}{| p{0.18\linewidth} | p{0.1\linewidth} | p{0.08\linewidth} | p{0.12\linewidth} | p{0.09\linewidth} | p{0.12\linewidth} | p{0.12\linewidth} |} \hline
    CVE & Rootless mode & User ID & Dropped Privileges & seccomp & AppArmor, SELinux & Alternative Runtimes \\ \hline
    CVE-2016-5195 & - & - & - & \cellcolor{yellow!25} +/- \linebreak with ptrace exploit & - & \cellcolor{green!25} + \\ \hline
    \end{tabular}
    \caption{CVE-2016-5195}
    \label{tab:h:9}
\end{table}



\FloatBarrier
\clearpage


\subsection{Outcomes}

In this section, the results from previous discussion will be combined (Table \ref{tab:h:10}).

\begin{table}[H]
    \centering \small
    \begin{tabular}{| p{0.18\linewidth} | p{0.1\linewidth} | p{0.08\linewidth} | p{0.12\linewidth} | p{0.09\linewidth} | p{0.12\linewidth} | p{0.12\linewidth} |} \hline
    CVE & Rootless mode & User ID & Dropped Privileges & seccomp & AppArmor, SELinux & Alternative Runtimes \\ \hline
    CVE-2022-0847 & - & - & - & \cellcolor{green!25} + & - & \cellcolor{green!25} +  \\ \hline
    CVE-2022-0492 & \cellcolor{green!25} + & \cellcolor{green!25} + & \cellcolor{yellow!25} +/- \linebreak if no userns & \cellcolor{green!25} + & \cellcolor{green!25} + & \cellcolor{green!25} + \\ \hline
    CVE-2022-0185 & - & - & \cellcolor{yellow!25} +/- \linebreak if no userns & \cellcolor{green!25} + &  & \cellcolor{green!25} + \\ \hline
    CVE-2021-3490 & - & - & \cellcolor{yellow!25} +/- \linebreak if no userns & \cellcolor{green!25} + &  & \cellcolor{green!25} + \\ \hline
    CVE-2021-31440 & - & - & \cellcolor{yellow!25} +/- \linebreak if no userns & \cellcolor{green!25} + &  & \cellcolor{green!25} + \\ \hline
    CVE-2021-22555 & - & - & \cellcolor{yellow!25} +/- \linebreak if no userns & \cellcolor{green!25} + &  & \cellcolor{green!25} + \\ \hline
    CVE-2017-1000112 & - & - & \cellcolor{yellow!25} +/- \linebreak if no userns & \cellcolor{green!25} + &  & \cellcolor{green!25} + \\ \hline
    CVE-2017-5123 & - & - & - & - & - & \cellcolor{green!25} + \\ \hline
    CVE-2016-5195 & - & - & - & \cellcolor{yellow!25} +/- \linebreak with ptrace exploit & - & \cellcolor{green!25} + \\ \hline
\end{tabular}
    \caption{Linux kernel vulnerabilities}
    \label{tab:h:10}
\end{table}

As can be seen, kernel vulnerabilities are particularly dangerous as they can often be exploited despite hardening measures. This is understandable, as most security features operate in the kernel and are unable to control malicious activities within the system.

Nevertheless, some security measures have demonstrated their effectiveness. For example, seccomp filtering is a tool that protects the system from almost all known kernel vulnerabilities. In addition, dropping unnecessary privileges also adds an extra layer of security.

Another way to install additional protection in sensitive and dangerous environments is to use alternative container runtimes. These tools, including Kata Containers, Unikernels, and Firecracker, use virtual machine isolation to run containers and are not affected by vulnerabilities in the host operating system's kernel. However, like any software, these tools can have their own vulnerabilities. As we mentioned earlier, there are some examples of this.

Overall, it is clear that a robust defense is essential for securing containerized applications. By doing so, it is possible to not only defend against attacks that exploit known vulnerabilities, but also protect the services from zero-day vulnerabilities that have not yet been reported.
